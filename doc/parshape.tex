

\load[mte]\enablemte
\enlang
\enquotes
\typosize[11/13]
\margins/1 a4 (2,2,2,2)cm
\nopagenumbers
\parindent=0pt
\parskip=\medskipamount

\def\linkblue{\setrgbcolor{0.06274509803921569 0.14901960784313725 0.5803921568627451}}
\hyperlinks \linkblue \linkblue


\fontdef\titlefont{\setfontsize{mag2.1}\bf}
\fontdef\subtitlefont{\setfontsize{mag1.3}\bf}

\def\center#1{\hfil{#1}\hfil}

\eoldef\title#1{\par
  \center{\titlefont #1}% \hfil vycentruje
  \medskip 					 % mezera po nadpisu
  \par\nobreak
}

\eoldef\subtitle#1{\par
	\noindent
	\medskip
	{\subtitlefont #1}
	\smallskip
}

\eoldef\curvetitle#1{
	{\setfontsize{mag1.1}\bf #1}
}

\def\macro#1{{\tt\char`\\#1}}

\def\ang#1{#1^{\circ}}

\def\exampleparshape{
\parshape 29
  46.393pt 422.434pt
  116.499pt 323.124pt
  171.071pt 253.747pt
  211.346pt 203.548pt
  238.668pt 169.32pt
  254.511pt 148.836pt
  260.478pt 140.126pt
  258.214pt 141.366pt
  249.514pt 150.701pt
  236.001pt 166.545pt
  219.163pt 187.557pt
  200.283pt 212.767pt
  180.429pt 241.737pt
  160.414pt 275.117pt
  140.853pt 317.797pt
  122.161pt 361.535pt
  104.627pt 358.131pt
  88.439pt 353.534pt
  73.68pt 347.509pt
  60.404pt 340.001pt
  48.573pt 331.047pt
  38.164pt 320.671pt
  29.139pt 339.875pt
  21.445pt 368.353pt
  14.99pt 395.593pt
  9.716pt 421.652pt
  5.562pt 446.59pt
  2.462pt 470.475pt
  0pt \hsize
\noindent
}


\def\loremonepar{%
Lorem ipsum dolor sit amet, consectetur adipiscing elit. Duis ullamcorper enim et dignissim feugiat. Cras blandit tincidunt augue, id maximus velit venenatis vel. Nam volutpat arcu in dignissim varius. Integer faucibus tristique ex, in egestas lectus posuere in. Maecenas at tellus tortor. Sed euismod id lorem dictum tincidunt. Aenean lobortis non tellus a efficitur. Aliquam id lectus sit amet ex tincidunt euismod. Quisque a condimentum arcu. Donec et dignissim quam. Praesent hendrerit arcu sit amet ultrices laoreet.
Quisque eu ex ut ante suscipit porttitor. Mauris erat libero, pulvinar non ante sed, ultrices fringilla purus. Praesent lacus ligula, eleifend vel urna ut, dictum porta metus. Nunc a urna eu ipsum ultrices commodo id non purus. Fusce volutpat auctor dolor, ut fringilla risus ultrices eget. Duis a aliquet dui. Nunc tempus auctor dapibus. Aliquam aliquam, metus at blandit scelerisque, elit tellus sodales ligula, at auctor tellus urna nec justo. Donec ullamcorper luctus nulla. Nunc rutrum, nulla id ultrices molestie, quam ex placerat nisi, ac consequat nisi urna lacinia ante. Ut pharetra aliquam lacus et dignissim. Quisque feugiat tristique maximus.
Vivamus eget auctor diam. Proin consectetur ante vel sapien hendrerit finibus. Nunc mattis erat vitae sem eleifend, varius tempus odio sollicitudin. Lorem ipsum dolor sit amet, consectetur adipiscing elit. Cras massa odio, dictum eget hendrerit ut, eleifend quis nulla. Aenean vehicula, nunc vel ullamcorper scelerisque, magna urna feugiat purus, vel facilisis eros dui ac neque. Nam eget mollis risus, ac gravida nunc. Suspendisse fermentum nibh ex, non suscipit mauris placerat id. Nulla lobortis risus neque, non euismod neque posuere vel. Praesent a diam in nisi efficitur pulvinar. Suspendisse cursus leo nec felis ultrices, et tempor tortor maximus. Duis tincidunt ultricies magna non suscipit. Phasellus interdum dapibus varius. Aliquam pellentesque elit diam, vel molestie lectus condimentum vel.
}


%%% obsah

\title Visual \char`\\parshape editor for \TeX

To try out the project, go to \ulink[https://parshape.com]{parshape.com} or checkout the
code at \ulink[https://github.com/tmscer/parshape]{github.com/tmscer/parshape}.
See this file's source at \ulink[https://github.com/tmscer/parshape/blob/master/doc/parshape.tex]{github.com/tmscer/parshape/blob/master/pages/index.js}.

\subtitle The problem

\macro{parshape} allows a {\TeX} user to shape their paragraph.
The following block of text was shaped using
\macro{parshape 3 2cm 12cm 4cm 320pt 2cm 12cm}:

\parshape 3
	2cm 12cm
	4cm 320pt
	2cm 12cm
\noindent
Notice the first line is moved by {\bf2cm} to the right and is {\bf12cm} long.
The second line is moved by {\bf4cm} to the right and is {\bf400pt} long.
The third line is move by {\bf2cm} to the right and is {\bf12cm} long.

Creating just these 3 lines took the author quite a while.
Trying to fit a paragraph around a more complex shape like an image, table or
some weird shape is cumbersome and time consuming. Other software like MS Word
help the user by being a visual tool where the user gets immediate feedback.
{\TeX}, of course, isn't that.

\subtitle The solution

Below you can see two versions of the {\tt<Canvas>} component of the web-based solution.

\hfil
\hbox{\picw=8cm \inspic{before-bezier.png} \picw=8cm \inspic{after-bezier.png}}
\hfil

On the left is the {\it before} state. After the user {\it right clicks} at the bottom end
of the blue Bézier curve. The new state shown on the right will emerge. A similar edit can be
made to the right edge of every line of the paragraph and with other geometrical shapes.

Here's how it looks like in use with some extra edits to the right edge of the paragraph:

\exampleparshape
\loremonepar

\subtitle Supported geometrical shapes and their math

The editor currently supports three geometrical shapes: lines, half-circles and Bézier curves.
Line has a variant called {\it snap-angle line} which forces the line's angle to be
a multiple of $15^{\circ}$.
Intersection with paragraph lines is always calculated for the middle of the paragraph
according to primitive register \macro{baselineskip}.

The coordinate system has its origin in the left top corner of where the text can start.
To the right x-axis increases and downward the y-axis also increases.

\curvetitle Line

This is the most simple geometric shape, Given two points, it is possible to get an equation
of the form $y = ax + b,\;a,\!b \in {\bbchar R}$. Then for every paragraph line, which are all
horizontal of the form $y = baselineskip \cdot (i + 0.5)$, we can solve for $x$
provided the drawn line isn't vertical. In all cases bounding box of the drawn line is checked
in order to avoid the change of paragraph lines outside it.

\curvetitle Snap-angle line

The equation remains the same as for the normal line. However, we have recalculate the second placed
point such that the line's angle is a multiple of $\ang{15}$ degrees or $\pi \over 12$ radians.
We calculate the line's angle using using $\theta = \tan^{-1}{a}$ and round it.
After that we {\it rotate} the second placed point using $\cos$ for x-coordinates $\sin$
for y-coordiantes.

\curvetitle Bézier curve

For $n+1$ points, coordinates of {\it some} point on the Bézier curve would be:

$$
\vec{B_t} =
t^0 (1-t)^n \cdot {\bf P_0} +
t^1 (1-t)^{n-1} {n \choose 1} \cdot {\bf P_1} +
\cdots +
t^k (1-t)^{n-k} {n \choose k} \cdot {\bf P_k} +
\cdots +
t^n (1-t)^0 {\bf P_n}
$$

where $\bf P_i$ are placed points and $k \in {\bbchar N}_0$ is an iterator from $0$ to $n$.
Parameter $t$ is unique for each point and it is from the interval $[0,1]$.

Intersection algorithm for Bézier curves does the following: for every paragraph
line, intersect with all lines of the Bézier curve approximation and choose the
leftmost or rightmost based on whether we're shaping the left edge or the right edge
of the paragraph. Essentially we're reusing the previous algorithm for lines several
times.

\curvetitle Half-circles

There are four cases to handle:
1) we're shaping the {\it left} edge with the {\it left} half a full circle,
2) we're shaping the {\it left} edge with the {\it right} a full circle,
3) we're shaping the {\it right} edge with the {\it left} half a full circle,
4) we're shaping the {\it right} edge with the {\it right} half a full circle.
That can be achieved by knowing how to intersect a full circle with a horizontal
paragraph line and the choosing which edge of the paragraph to change and choosing
the intersection on left or the right.

Equation for a circle with an origin $(x_0, y_0)$ has the form

$$
(x - x_0)^2 + (y - y_0)^2 = r^2
$$

where $r$ stands for radius. Given a paragraph line $y = c,\;c\in{\bbchar R}$, we can solve
for $x$:

$$
\eqalignno{
(x - x_0)^2 + (c - y_0)^2 &= r^2 \cr
x^2 -2x_0 + x_0^2+ (c - y_0)^2 &= r^2 \cr
x^2 -2x_0x + (x_0^2+ (c - y_0)^2 - r^2) &= 0
}
$$

That's nothing more than an ordinary quadratic equation. When determinant $D < 0$,
the circle has no intersection with the current paragraph line. When determinant $D = 0$,
there is only one intersection. When determinant $D > 0$, there are two intersections
and we have to choose based on whether we're using a left or a right half-circle.

\bye
